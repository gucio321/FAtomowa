\section{Wykład 09.04.2025}

\subsection{Cząstka w polu Elektromagnetycznym}

Rozważamy \textbf{Hamiltonian klasyczny} cząstki z ładunkiem $q$ w pol elektromagnetycznym (dalej EM) $(A, \phi)$:

$$
\vec{H} = \frac{\left(\vec{p} - q\vec{A}\right)^2}{2m} + q\phi
$$
Pęd Uogulniony z kolei przedstawiamy jako $\vec{p} = m\vec{v} + q\vec{A}$.

Następnie przechodzimy na zapis kwantowy, czyli $\vec{H} \to \hat{H}$.
W takiej sytuacji, $\vec{p} \to \hat{p} = i \hbar \nabla$

$$
\hat{H} = \frac{\left (\hat{p} + q\vec{A}\right)^2}{2m} + q\phi = \frac{1}{2m}\left(\hat{p}^2 + q\hat{p}\cdot\vec{A} + q\vec{A}\cdot\hat{p} + q^2\vec{A}^2\right) + q\phi
$$

W ten sposób otrzymujemy tak zwane \textbf{Przybliżenie Półklasyczne}.

\begin{tcolorbox}
        \textbf{Przybliżenie Półklasyczne} zakłada cząstkę kwantową, oraz
        pole $\approx$ pole EM, które da się (jeszcze) opiswyać klasycznie.
\end{tcolorbox}

\end{document}
